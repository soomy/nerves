\chapter{Outlook} \label{chap:outlook}
The developed method segments the sciatic nerve with human-level performance regarding quantitative evaluation metrics. As we aim to leverage the diagnosis and the assessment of all peripheral nerves, obviously future work should go in the direction of a method that can segment all peripheral nerves. Although not used in this work, we have further \gls{mrn} images available from different anatomical regions (upper \& lower arm, lower leg) for the volunteer cohort and also the patients in the registry. A method could be developed incorporating these \gls{mrn} images. However, given that the sciatic nerve is the largest nerve in the human body, segmenting smaller peripheral nerves might be difficult. Almost certainly a large number of \gls{mrn} images with higher resolution would be required to achieve an acceptable level of segmentation performance. Once peripheral nerves can robustly be segmented, potentially new biomarkers, such as texture analysis-based biomarkers~\cite{FelisazTextureNeuropathy}, could be extracted from the \gls{mrn} images and used for assessment and diagnosis of neuropathies. However, the development of biomarkers might require matched cohorts to be clinically meaningful. Furthermore, an expert assessment in the form of an outlier-analysis could be an advisable next step to do on our \gls{mrn} images. As we showed in the first experiment, the trained neural networks mainly relying on the \gls{t2} images. However, the \gls{ir} images reduced false positive outliers and helped the network to distinguish between nerve and blood vessels. Therefore, it is crucial to further refine and optimize the \gls{mrn} sequences for even better imaging of the nerves. Both, clinicians and methods like the one we proposed would be the direct benefactors.

Regarding technical developments, a point could be to reduce the axial spacing during the \gls{mrn} acquisition (e.g., \gls{3d} \gls{mrn} sequences) and reinvestigate the potential segmentation performance increase of \gls{3d} context.
Furthermore, the impact of the projection-based loss could be examined in more detail because it almost achieved the same results as the cross-entropy loss. Perhaps more axial slices could be used to calculate the projections. Alternatively, a stack-wise 5-to-1 architecture could be expanded by two additional loss terms based on the sagittal and coronal projections. The projection-loss could be a way to incorporate a possible anatomical tubular-like shape prior to the learning phase of a neural network. Furthermore, we think that a combination of a neural network in conjunction with post-processing is the future method of peripheral nerve segmentation. Post-processing, however, could be more sophisticated than ours. Our post-processing method could be improved by taking multiple paths through the graph, in order to account for the branching of the nerve. One could also change to a more sophisticated method based on a probabilistic model with anatomical constraints, similarly to~\cite{Rempfler2015ReconstructingProgramming}.
\endinput